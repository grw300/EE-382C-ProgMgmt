\documentclass{article}
\usepackage[margin=3.5cm]{geometry}
\usepackage{pdfpages}
\usepackage{graphicx}
\usepackage{booktabs}
\usepackage{float}
\usepackage{color}
\usepackage{amsmath}
\DeclareGraphicsExtensions{.pdf,.png,.jpg}	

\newcommand{\ra}[1]{\renewcommand{\arraystretch}{#1}}

\title{Homework IV}
\author{Gregory Williams\\GW4975\\EE 382C Requirements Engineering}
\date{12/03/2015}

\begin{document}
	\maketitle
	
	\section*{1}
	\subsection*{(a)}

	\begin{center}
	\begin{table*}[ht!]
	\centering
	\ra{1.3}
	\begin{tabular}{@{}lr@{}} \toprule
	Item & \multicolumn{1}{c}{\$} \\
	\midrule
	Revenue & 18000\\
	COGS 	& \color{red}6750\\  \midrule
	\textbf{Gross Margin} & 11250\\
	G\&A Costs & \color{red}6000\\
	Depreciation & \color{red}750\\\midrule
	\textbf{Net Income} & 4500\\ 
	Income Tax (@ 30\%) & \color{red}1350\\ \midrule
	\textbf{Net Income After Tax} & \textbf{3150}
	\end{tabular}
	\caption{Net Income After Tax}
	\end{table*}
	\end{center}
	
	\pagebreak
	
	\subsection*{(b)}

	\begin{center}
	\begin{table*}[ht!]
	\centering
	\ra{1.3}
	\begin{tabular}{@{}lrlr@{}} \toprule
	\textbf{Assests} && \textbf{Liabilities} & \\
	\cmidrule{1-1}\cmidrule{3-3}
	Cash & 15650 & Loan & 10000\\
	Equipment 	& 7000 & \textbf{Total Liabilities} & 10000\\
	Inventory & 2500 && \\
	AR & 3000 & \textbf{Equity} &\\ \cmidrule{3-3}
	&& Initial & 15000\\
	&& Retained Earnings & 3150\\ 
	&& \textbf{Total Equity} & 18500\\\midrule
	&& \textbf{Total Liabilities +} &\\
	\textbf{Total Assests} & 28150 & \textbf{Equity} & 28150\\
	\end{tabular}
	\caption{Balance Sheet}
	\end{table*}
	\end{center}

	\subsection*{(c)}

	\begin{center}
	\begin{table*}[ht!]
	\centering
	\ra{1.3}
	\begin{tabular}{@{}lr@{}} \toprule
	Flows & \multicolumn{1}{c}{\$} \\
	\midrule
	Inward & \\ \cmidrule{1-1}
	Loan & 10000\\
	Collected Sales 	& 15000\\  \midrule
	\textbf{In} & 25000\\
	Outward & \\ \cmidrule{1-1}
	Equipment & 8000\\
	Product (2500 @ 4.50) & 9000\\
	G\&A & 6000\\ 
	Tax & 1350\\ \midrule
	\textbf{Out} & 24350\\ 
	\textbf{Net} & 650
	\end{tabular}
	\caption{Cash Flow}
	\end{table*}
	\end{center}
	
	\section*{2}
	\subsection*{(a)}
	\begin{align*}
	PW &= 5000 \\
	&+ 1000 [(P/F, 9\%, 1)+ (P/F, 9\%, 2) + (P/F, 9\%, 3)] \\
	&+ 3000 [(P/F, 9\%, 4) + (P/F, 9\%, 5) + (P/F, 9\%, 6) + (P/F, 9\%, 7) + (P/F, 9\%, 8)] \\
	&= 5000 + 1000 (2.53129) + 3000 (3.00353) \\
	&= 16541.88
	\end{align*}

	\subsection*{(b)}
	\begin{align*}
		FW &= 5000 (F/P, 9\%, 8) \\ 
		&+ 1000 [(F/P, 9\%, 7) + (F/P, 9\%, 6) + (F/P, 9\%, 5)] \\
		&+ 3000 [(F/P, 9\%, 4) + (F/P, 9\%, 3) + (F/P, 9\%, 2) + (F/P, 9\%, 1) + (F/P, 9\%, 0)] \\
		&= 5000 (1.99256) + 1000 (5.04376) + 3000 (5.98471) \\
		&= 32960.69 \\
	\end{align*}
	
	\section*{3}
	\begin{align*}
	F/A &= (F/A, 9\%, 18) = 41.30134 \\
41.30134 A &= 20,000 [1 + (P/A, 9\%, 3)] 
	\\41.30134 A &= 20,000 (3.53129) \\
	A &= 1710.01
	\end{align*}

	\section*{4}
	\begin{align*}
	200,000/25,000 &= 8 yrs\\
	NPW(9\%) &= \sum_{t=0}^{N}\frac{R_t}{(1+i)^t}\\
	&= -200000 + 22935.78 + 21042.00 + 19304.59 \\
	&+ 17710.63 + 16248.28 + 14906.68 + 13675.86 \\
	&+ 12546.66 + 11510.69 + 10560.27 + 9688.32 \\
	&+ 8888.37 + 8154.47 + 7481.16 + 6863.45 \\
	&= -200000 + 201517.21\\
	&= 1517.21
	\end{align*}
	
	\section*{5}
	
	\begin{align*}
	F/P &= 30000/12000  \\
	&= 2.5\\
IRR &= (F/P)^{1/6} - 1\\
	&= 2.5^{1/6} -1 \\
	&= 0.165 \rightarrow 16.5\%\\
	NPW(9\%) &= \sum_{t=0}^{N}\frac{R_t}{(1+i)^t}\\
	&= -200000 + 0 + 0 + 0 \\
	&+ 0 + 0 + 17888.02 \\
	&= -12000 + 17888.02\\
	&= 5888.02
	\end{align*}
	
	\section*{(6)}
	\noindent We first find the $AEW$ for $A$:
	\begin{align*}
		AEW_{A} &= 600000 (A/P, 12\%, 3) - 320000 \\
		&= 600000 (0.41635) - 320000 \\
		&= -70190/yr
	\intertext{\noindent We then find the $AEW$ for $B$:}
		AEW_{B} &= 1000000 (A/P, 12\%, 4) - 200000 (A/F, 12\%, 4) - 350000 \\
		&= 1000000(0.32923) - 200000 (0.20923) - 350000 \\
		&= -62616/yr
	\end{align*}
	Thus choose A, since it will save the company more.
\end{document}